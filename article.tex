\documentclass[10pt, a4paper,spanish]{article}
\usepackage[utf8]{inputenc}

\usepackage{lipsum} % Package to generate dummy text throughout this template
\usepackage{varwidth}
\usepackage{hyperref}
\usepackage{graphicx}

\usepackage[T1]{fontenc} % Use 8-bit encoding that has 256 glyphs
\usepackage{microtype} % Slightly tweak font spacing for aesthetics

\usepackage[hmarginratio=1:1,top=32mm,columnsep=20pt]{geometry} % Document margins
\usepackage[hang, small,labelfont=bf,up,textfont=it,up]{caption} % Custom captions under/above floats in tables or figures
\usepackage{booktabs} % Horizontal rules in tables
\usepackage{float} % Required for tables and figures in the multi-column environment - they need to be placed in specific locations with the [H] (e.g. \begin{table}[H])
\usepackage{hyperref} % For hyperlinks in the PDF

\usepackage{lettrine} % The lettrine is the first enlarged letter at the beginning of the text
\usepackage{paralist} % Used for the compactitem environment which makes bullet points with less space between them

\usepackage{abstract} % Allows abstract customization
\renewcommand{\abstractnamefont}{\normalfont\bfseries} % Set the "Abstract" text to bold
\renewcommand{\abstracttextfont}{\normalfont\small\itshape} % Set the abstract itself to small italic text

\usepackage{titlesec} % Allows customization of titles
\renewcommand\thesection{\Roman{section}} % Roman numerals for the sections
\renewcommand\thesubsection{\Roman{subsection}} % Roman numerals for subsections
\titleformat{\section}[block]{\large\scshape\centering}{\thesection.}{1em}{} % Change the look of the section titles
\titleformat{\subsection}[block]{\large}{\thesubsection.}{1em}{} % Change the look of the section titles

\usepackage{fancyhdr} % Headers and footers
\pagestyle{fancy} % All pages have headers and footers
\fancyhead{} % Blank out the default header
\fancyfoot{} % Blank out the default footer
\fancyhead[C]{ \today $\bullet$ Definición de diversos ataques de seguridad} % Custom header text
\fancyfoot[RO,LE]{\thepage} % Custom footer text

%----------------------------------------------------------------------------------------
%	TITLE SECTION
%----------------------------------------------------------------------------------------

\title{\vspace{-15mm}\fontsize{24pt}{10pt}\selectfont\textbf{Definición de diversos ataques de seguridad}} % Article title

\author{Sergio García Prado}
\date{\today}

%----------------------------------------------------------------------------------------

\begin{document}

	\maketitle % Insert title
	\thispagestyle{fancy} % All pages have headers and footers


%----------------------------------------------------------------------------------------
%	TEXT
%----------------------------------------------------------------------------------------

    \section{Brute Force: Fuerza Bruta}
        \paragraph{}
		El ataque por fuerza bruta se basa en la obtención de la clave para acceder a un recurso protegido, ya sea el acceso a un servicio o el descifrado de datos. El método consiste en probar diferentes claves hasta encontrar la correcta. Para ello se suelen utilizar diccionarios de claves más comunes. Este método se suele utilizar cuando no existe otro que pueda aprovechar alguna vulnerabilidad del sistema objetivo. Computacionalmente es muy costoso ya que en el peor caso tendría que probar todas las posibles combinaciones. En muchos casos es fácil protegerse ante este tipo de ataques limitando el número de intentos al introducir la clave.


    \section{Cache Poisoning: Envenenamiento de Caché}
        \paragraph{}
		Este tipo de ataque consiste en la substitución del contenido de una determinada cache por otro de carácter malicioso que será servido como si fuera el original. Este ataque normalmente se realiza sobre caches web afectando a todos los usuarios que utilicen dicha cache.


    \section{DNS Poisoning: Envenenamiento de DNS}
        \paragraph{}
		El envenenamiento de DNS sigue la misma estrategia que el ataque anterior, sólo que en este caso no se orienta hacia una cache sino hacia un servidor DNS. Con esto se consigue redirigir el tráfico dirigido a un determinado host hacia otro con intenciones poco claras. Generalmente estos destinos simulan el comportamiento que tenía el antiguo servidor pero además tratan de extraer claves u otros recursos valiosos.


    \section{Cross-Site Request Forgery (CSRF) o Falsificación de petición en sitios cruzados}
        \paragraph{}


    \section{Cross-Site Scripting (XSS) o Secuencias de comandos en sitios cruzados}
        \paragraph{}


    \section{Denial of Service (DoS)}
        \paragraph{}


    \section{LDAP injection}
        \paragraph{}


    \section{Man-in-the-middle}
        \paragraph{}


    \section{Session hijacking attack}
        \paragraph{}


    \section{SQL Injection: Inyección SQL}
        \paragraph{}


\end{document}
